\section{Questão 2}

  Para a realização da questão 2 foram gerados dois programas. Ambos realizam o cálculo do valor máximo de uma sequência de inteiros. O primeiro chamado \textit{main} utiliza-se de \textit{threads} enquanto o segundo chamado \textit{analise} foi implementado da maneira tradicional.


\subsection{Primerio programa - main}

  Para gerar o executável do primeiro programa, \textit{main}, rode no terminal o comando: make main. Ele ira produzir o arquivo \textit{max}.

  No terminal, execute o programa \textit{./max} e informe: o tamanho da lista e a sequência de inteiros. Ficando assim:

  \begin{itemize}
    \item ./max tamanhoLista inteiro1 inteiro2 ... tamanhoLista-1
  \end{itemize}

  \subsubsection{Caso de Teste: Valores válidos}

    Supondo que o usuário informe a seguinte entrada:

    \begin{itemize}
      \item ./max 4 3 2 7 9
    \end{itemize}

    o sistema irá imprimir:

    \begin{itemize}
      \item O tamanho da lista
      \item Os valores da lista
      \item O vetor W após a inicialização
      \item O vetor W após o passo 2
      \item O valor máximo
      \item E a posição do valor máximo
    \end{itemize}

    Resultado:


      Number of input values = 4

      Input values x = 3 2 7 9

      After initialization w = 1 1 1 1

      After Step 2

      w = 0 0 0 1

      Maximum = 9

      Location = 3



  \subsubsection{Caso de Teste: Valores inválidos}

  Caso o usuário informe um número maior que 100 para o tamanho da lista, ou ainda, informe uma sequência de números maior que o tamanho da lista, o programa informa uma mensagem de erro e encerra o programa.


\subsection{Segundo programa - analise}

  Gere o executável para o segundo programa com o comando: make analise. Ele irá produzir o arquivo compilado \textit{alanise}.

  Ele é executado da mesma forma que o primeiro programa, ficando assim:

  \begin{itemize}
    \item ./analise 4 3 2 7 9
  \end{itemize}

  O resultado produzido será:

  Number of input values = 4
  Input values x = 2 3 7 9
  Max value: 9

\subsection{Questão de Análise}

  \textbf{QUESTÃO: Por que precisamos n(n-1)/2 em vez de n*n \textit{threads}?}

  Gerando \textbf{n(n-1)/2} \textit{threads} permite criar um número suficiente de \textit{threads} capaz de realizar a comparação de dois inteiros de toda a lista apenas uma vez.

  \textbf{Comparação dos dois programas}

  Para realizar a comparação dos dois programas foi utilizada a função \textit{time} do próprio sistema operacional. Ela fornece o tempo de execução do programa informado.

  Para utilizá-la informe \textit{time} no momento da execução do programa, da seguinte forma:

  \begin{itemize}
    \item time ./max 4 3 2 7 9
    \item time ./analise 4 3 2 7 9
  \end{itemize}

  O programa realizado com \textit{threads} demorou 21 segundos, enquanto o programa tradicional levou apenas 19 segundos.

  Ao implementarmos programas com \textit{threads} temos a errônea impressão que consequentemente ele será mais rápido. Contudo, foram criadas várias \textit{threads} para realizar uma operação simples. Tais \textit{threads} consomem mais recursos do sistema, e por isso levam um tempo maior para executar a mesma operação.

  \subsection{Inconsistências identificadas}

  Na etapa 2 do primeiro programa, \textit{main}, mais especificamente no método: \textbf{comparing\_threads} foi gerado um número \textbf{n(n-1)/2} de \textit{threads}, conforme requisitado na questão. No entanto, para realizar a comparação entre os números da lista, foi necessário utilizar duas estruturas \textit{for}. Dessa forma, para uma lista de tamanho 4, será preciso 6 \textit{threads}, porém, da maneira que foi implementado, cada uma realiza 6 comparações, produzindo um total de 36 comparações.
