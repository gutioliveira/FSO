\chapter[Introdução]{Introdução}

	Este trabalho tem como objetivo consolidar os conhecimentos adquiridos na disciplina de Fundamentos de Sistemas
	Operacionais ofertada na Universidade de Brasília, Faculdade Gama.

\section{Ambiente de Desenvolvimento}

	Para uma melhor realização do trabalho, decidiu-se por unificar, entre a dupla, as ferramentas de desenvolvimento
	utilizadas, as quais incluem sistema operacional, compilador, depurador e editor de texto. Portanto, o seguinte
	ambiente de desenvolvimento foi estabelecido:

  A configuração do ambiente de desenvolvimento utilizado para a condução desse trabalho é listada a seguir:

	\begin{itemize}

		\item O sistema operacional utilizado foi o \textit{Linux} na distro Elementary OS 0.3.2 Freya LTS;
		\item O código foi convertido com o compilador \textit{GCC} na versão 4.8.4;
		\item O código foi depurado, quando necessário, com o debugger \textit{GDB} na versão 7.7.1;
		\item A edição dos arquivos de código e texto foi realizada com o \textit{Sublime}.

	\end{itemize}
