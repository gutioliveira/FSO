\chapter[Questões]{Questões}

\begin{itemize}
	\item Em relação ao programa que contempla os itens a) e b), quais foram as alterações
	de códigos-fonte necessárias para a solução (se houverem)?
\end{itemize}

Não houve alterações nos códigos fonte das bibliotecas, somente nos arquivos Makefile.

\begin{itemize}
	\item Dados os conhecimentos adquiridos em função desse trabalho, indique vantagens e problemas decorrentes da utilização de bibliotecas dinâmicas.

	\begin{itemize}

	\item Vantagens:

	Salva espaço em disco, caso n programas utilizem a biblioteca, não será necessário n cópias da biblioteca para funcionamento dos programas.
	Pode-se evoluir a biblioteca sem necessariamente ter que evoluir os programas que a utilizam, podendo-se corrigir bugs da biblioteca sem mecher nos códigos fonte desses programas.

	\item Problemas:

	A vantagem de atualizar uma biblioteca sem necessariamente mecher nos programas que a utilizam pode ser uma desvantagem também, caso a atualização de uma biblioteca quebre os programas que a utilizam. 

	Outra desvantagem seria a instalação da biblioteca, não é possível instalar a biblioteca em /lib ou /usr/lib sem permissão de administrador, caso não seja possível instalar nesses diretórios, será necessário deixar explícito o local onde a biblioteca ficará, gerando um passo adicional aos usuários.

	\end{itemize}

\end{itemize}

% \item Dados os conhecimentos adquiridos em função desse trabalho, indique vantagens e problemas decorrentes da utilização de bibliotecas dinâmicas.



