\section{Questão 3}

  Para implementar a questão 3 foram gerados os programas q03a e q03b, o programa q03c não foi implementado. O programa q03a realiza o cálculo do produto entre duas matrizes utilizando apenas uma \textit{thread}, enquanto o programa q03b realiza o mesmo cálculo com um maior número de \textit{threads}.

  Para gerar o arquivo executável da questão q03a, digite no terminal: make qa. Que irá gerar o executável \textit{q03a}. Já para a questão q03b, rode o comando: make qb, gerando o arquivo \textit{q03b}.

  Para rodar os programas gerados informe o nome do executável mais o arquivo de entrada \textit{int.txt}. Ficando da seguinte maneira:

  \begin{itemize}
    \item ./q03a < in.txt
    \item ./q03b < in.txt
  \end{itemize}

  Ambos irão produzir o mesmo resultado.

  \subsection{Caso de Teste: Valores válidos}

    Foi fornecido um arquivo de entrada chamado \textit{in.txt} contendo entradas válidas. Nele é informado valores referentes a duas matrizes, sendo a primeria matriz 3x2 e a segunda 2x3.

    Os programas irão calcular o produto de duas matrizes e imprimir os valores referentes a matriz resultado e as matrizes informadas. Ficando da seguinte maneira:


Dados da matriz A:

1 4


2 5


3 6


Dados da matriz B:

7 9 11

8 10 12

Matriz resultante:

39 49 59

54 68 82

69 87 105

\subsection{Limitações}

Ambos os programas gerados não fazem verificação quanto a entrada de dados, assim sendo, podem gerar resultados inesperados. Sua maior limitação está no fato de que eles suportam apenas a multiplicação de matrizes (3x2)*(2x3).
