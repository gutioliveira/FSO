
	\section{Instruções}
		\subsection{Compilação}

	  Para compilar as questões: questão1a, questão1b e questão1c  foi utilizado um arquivo Makefile para cada uma delas, no qual contém as regras de compilação. Para gerar o executável entre no diretório referente a questão desejada e digite \textit{make} no terminal.

	  \subsection{Execução}

	  Para executar o programa de qualquer uma das questões digite \textit{./gera\_primo} no terminal.
	  Após execução do programa, será mostrado um menu em \textit{loop} com as seguintes opções:

	  \textbf{1) Gerar primo.}
	  \textbf{2) Testar primo.}
	  \textbf{0) Sair.}

	  Para entrar em uma das opções basta digitar o número correspondente na tela, sendo que na opção 2, será pedido um número e será exibido na tela se o mesmo é primo ou não.

  \section{Limitações}

  	\subsection{Entrada}

  		O menu do programa não valida nenhuma entrada inválida, e caso seja passado um número inteiro que não esteja nas opções do menu, o programa irá encerrar.
  		O programa só aceita números inteiros positivos com o limite da variável \textit{int} da linguagem C.

  	\subsection{Números pseudo aleatórios}

  		Para gerar os números pseudo aleatórios foi utilizado a função \textit{rand()} da biblioteca \textit{stdlib}. Como o intervalo de geração dos inteiros são todos os números de até 32 \textit{bits} e a função utiliza o tempo do computador como base, normalmente são gerados números muito grandes e caso o intervalo de geração dos números seja muito pequeno, pode ser gerado dois números iguais em seguida.


